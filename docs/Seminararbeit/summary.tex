chapter{Fazit und Ausblick}


Die Ziele der vorliegenden wissenschaftlichen Arbeit waren die Evaluierung geeigneter \ac{OCR}-Bibliotheken und die Entwicklung eines technischen Prototyps. Des Weiteren sollte eine geeignete Infrastruktur für die zukünftige Programmierung evaluiert und geschaffen werden. \\

\section{Fazit}

\subsubsection{Infrastruktur}
Durch die Installation der eingesetzten Infrastruktur wurden Build-Fehler sofort entdeckt und konnten behoben werden. Durch eine umfangreiche Testumgebung, die sowohl das Testen der einzelnen Softwarekomponenten, als auch das Zusammenspiel durch Integrationstests prüfte, konnten Fehler frühzeitig erkannt werden. Eingesetzte Tools, die das Testen vereinfachten, wurden in Kapitel \ref{testCases} dargestellt. Durch die Implementierung von Akzeptanztests konnte die erfolgreiche Umsetzung der gestellten Anforderungen nachgewiesen werden und die Abnahme des technischen Prototypen stellte kein Problem dar. Auch die nächtliche Analyse des Codes ermöglichte eine gute Qualität des programmierten Codes, sowie eine stets aktuelle Dokumentation.

\subsubsection{Regelbildung}
Während der gesamten Entwicklung wurde auf eine mögliche Weiterentwicklung des Codes geachtet. Angedachte Möglichkeiten die \ac{OCR}-Analyse nicht nur auf Lieferscheine zu begrenzen, sondern auch für weitere Einsatzgebiete zu nutzen, wurden bereits vorgesehen. Durch eine möglichst universal-gültige Regelbildung für die Klassifikation und Informationsextraktion, siehe Kapitel \ref{Classification} und Kapitel \ref{informationExtraction}, können neue Dokumententypen hinzugefügt werden. Auch die Entwicklung einer Trainingsanwendung, die aus zeitlichen Gründen keinen Platz in der vorliegenden Arbeit findet, kann durchgeführt werden. Diese Anwendung wird genutzt werden um die Regeln für die Klassifikation und Informationsextraktion dynamisch zu trainieren. Aktuell gültige Regeln können zur Bearbeitung abgefragt werden und sind nach erneutem Speichern für die nächsten \ac{OCR}-Analysen bereits gültig.

\subsubsection{Architektur}
Durch die implementierte verteilte Architektur ist ein mögliches Problem in der Geschwindigkeit der Anwendung vernachlässigbar. In Zeiten höherer Auslastung, beispielsweise morgens wenn neue Lieferscheine bereitstehen, können weitere Rechenkapazitäten installiert werden. Zukünftig wäre eine automatische, dynamische Skalierung der Anwendung möglich. Es könnte sein, dass der Master-Server, vgl. Kapitel \ref{OCRServer}, hohe Auslastungen bzw. lange Wartezeiten erkennt und anhand eines geeigneten Deployment-Verfahrens\footnote{{} mögliche Verfahren werden in Kapitel \ref{deployment} vorgestellt} neue Rechenmaschinen einspeist.\\
Durch die lose Kopplung der Komponenten ist eine Weiterentwicklung weiterer Eingabe Möglichkeiten für die Zukunft möglich. Einige Varianten wurden bereits in Kapitel \ref{inputPossibilities} vorgestellt. Sinnvoll wäre sicherlich per E-Mail geschickte Dokumente automatisch durch eine \ac{OCR}-Analyse zu klassifizieren und gegebenenfalls Informationen zu extrahieren. Abschließend könnten Archivierungsmaßnahmen durchgeführt werden.

\subsubsection{Komponenten}
Dadurch, dass der technische Prototyp auf vielen eingesetzten Bibliotheken arbeitet, bestehen einige Abhängigkeiten. Während der Entwicklung wurde darauf geachtet, dass Abhängigkeiten möglichst abstrahiert angesprochen werden können. Beispielsweise ist durch wenige Anpassungen ein Wechsel des MessageBusses möglich. Durch den Einsatz gängiger Design-Patterns, wie beispielsweise das Repository Pattern und die Implementierung einer Unit of Work (vgl. hier Kapitel \ref{repoPattern}), ist die Abstraktion zur verwendeten Datenbank gelungen. Zukünftig könnte der Einsatz einer dokumenten-basierten Datenbank in Betracht gezogen werden, da die Geschäftslogik der Anwendung nicht direkt mit der Datenbank kommuniziert. Im Falle eines Wechsels wäre nur die abstrahierte Logik der Repositories anzupassen. Auch die größte Abhängigkeit, nämlich die zur verwendeten \ac{OCR}-Bibliothek, konnte entkräftet werden. Durch die anpassbare Analyse der Dokumentenstruktur, mittels des implementierten \ac{XML} Serializers, muss eine neu eingesetzte Bibliothek nur das Ergebnis im \ac{XML}-Format zurückliefern können.\footnote{{} Die Analyse der Dokumentenstruktur ist in Kapitel \ref{documentStructureAnalyzer} beschrieben.} Die Implementierung der Klassifizierung und Informationsextraktion ist unabhängig von der eingesetzten \ac{OCR}-Bibliothek. Wie in Kapitel \ref{Classification} beschrieben, arbeitet die implementierte Klassifizierung anhand der analysierten Dokumenten-Struktur.\\
Die implementierten Schnittstellen konnten durch eingesetzte Tools dokumentiert und getestet werden.\footnote{{} Die Dokumentation der Schnittstelle wird in Kapitel \ref{apiDoc} aufgezeigt.} Mit Hilfe der analysierten Schnittstellen kann stets ein aktueller Client generiert werden um mit den Komponenten zu kommunizieren. Dieser Aspekt wurde in der implementierten Client-Bibliothek genutzt, damit stets ein testbarer Endpunkt für die Client-Kommunikation angeboten werden kann.\\

\section{Ausblick}

Zukünftig könnte das Trainingsprogramm für die dynamische Regelbildung durch den Enduser implementiert werden. Aufgrund einer möglichst generischen Regelbildung sind hier alle Vorbereitungen getroffen worden. So ist eine Korrektur bzw. Nachbesserung durch den Fachanwender direkt gegeben. Auch können leicht weitere Strategien für die Klassifizierung und Extraktion durch den Entwickler erfolgen, da entsprechende Design-Patterns umgesetzt sind.\\
Durch die Implementierung einer Pipeline beim Analysevorgang können einfach neue Filter hinzugefügt werden. Falls dieser Punkt zukünftig erweitert werden muss, könnte ein ganz anderer Ansatz umgesetzt werden. Denkbar wäre ein dynamischer Workflow-Ansatz, in dem definierte Schritte zu einem Workflow zusammengesetzt werden können. Dadurch ist eine flexiblere Zusammenstellung einzelner Komponenten gegeben.\\
Bei der eingesetzten Architektur könnte die Auswahl passender Rechenmaschinen für die Analyse, verfeinert werden. So könnten verschiedene Rechner mit unterschiedlichen Einstellungen der \ac{OCR}-Bibliothek gestartet werden. Am Master-Server, der für die Annahme und Verteilung der Aufträge zuständig ist, kann nun anhand der geforderten Einstellungen der Client-Bibliothek ein passender Slave-Rechner ausgesucht werden. Dies würde allerdings eine zusätzliche Kommunikation zwischen Master und Slave erfordern, da der Master wissen müsste, welche Einstellungen am Slave geladen wurden.\\
Das Logging könnte um ein aktives Monitoring erweitert werden, in dem der Zustand jeder Komponente zu jedem Zeitpunkt abgefragt wird. Diese Ergebnisse könnten in einem Dashboard zur besseren Übersicht dargestellt werden.

%Zum Abschluss möchte ich mich sehr herzlich bei den Personen bedanken, die mich während der gesamten Ausbildung und bei der Erstellung der vorliegenden Arbeit unterstützt haben. Besonderer Dank geht hier an meinen Betreuer Herr Sebastian Köhler, der mir stets mit Rat und Tat zur Seite stand.
%%TODO Schlusssatz finden